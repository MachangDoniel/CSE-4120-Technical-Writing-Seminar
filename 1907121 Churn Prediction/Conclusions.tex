\section*{}
\begin{center}
    {\fontsize{14}{1.5}\selectfont \textbf{CHAPTER VIII}}\\
    \vspace{12pt}
    {\fontsize{16}{1.5}\selectfont \textbf{Conclusions}}\\
    \vspace{12pt}
    \vspace{12pt}
\end{center}

\setcounter{section}{8}
\setcounter{subsection}{0}
\addcontentsline{toc}{section}{\textbf{CHAPTER VIII Conclusions}} % Add to ToC

\subsection{Summary} {
The combined research from these three papers demonstrates that machine learning models, particularly those involving neural networks and Naive Bayes algorithms, significantly outperform traditional methods like logistic regression in predicting customer churn. The use of hybrid and ensemble models, such as combinations of artificial neural networks (ANN), enhances prediction accuracy and reduces errors. Overall, these studies indicate that advanced machine learning techniques provide robust solutions for churn prediction across different datasets and contexts.
}
\subsection{Limitations} {
A common limitation across these studies is the reliance on specific datasets, which may not fully capture the diversity of customer behavior across various domains and industries. Additionally, while focusing on particular machine learning models (e.g., Naive Bayes, ANNs), the studies did not extensively explore other advanced algorithms, potentially limiting the comprehensiveness of their findings. The representativeness of the testing data and the generalizability of the results to real-world scenarios were also identified as concerns.
}

\subsection{Recommendations and Future Works} {
To improve the robustness and accuracy of churn prediction models, it is recommended to:

    Expand datasets to include more features and diverse customer behaviors from various industries.
    Incorporate a wider range of machine learning techniques, such as gradient boosting, support vector machines, and genetic algorithms.
    Enhance data preprocessing with advanced feature selection and dimensionality reduction methods.
    Integrate real-time data analysis and feedback mechanisms to improve model adaptability and performance in dynamic environments.

}