\section*{}
\begin{center}
    {\fontsize{14}{1.5}\selectfont \textbf{CHAPTER V}}\\
    \vspace{12pt}
    {\fontsize{16}{1.5}\selectfont \textbf{Findings and Recommendations}}\\
    \vspace{12pt}
    \vspace{12pt}
\end{center}

\setcounter{section}{5}
\setcounter{subsection}{0}
\addcontentsline{toc}{section}{\textbf{CHAPTER V Findings and Recommendations }} % Add to ToC

\subsection{Findings:}

    \subsubsection{Model Performance:} The evaluation metrics (accuracy, precision, recall, F1-score, AUC-ROC) will reveal the effectiveness of the models in predicting churn. You can identify the best performing model based on these metrics.
    \subsubsection{Feature Importance:} The analysis will highlight the customer behavior and service usage factors that have the most significant influence on churn. This provides valuable insights into customer behavior.

\subsection{Recommendations:}

    \subsubsection{Targeted Retention Strategies:} Use the churn prediction models to identify customers at high risk of churn. Develop targeted retention strategies for these segments, addressing the specific factors identified through feature importance analysis. This could include personalized offers, improved customer service, or loyalty programs.
    \subsubsection{Model Improvement:} Continuously monitor and improve the churn prediction models. Explore advanced techniques like deep learning or ensemble methods if the current models show room for improvement.
    \subsubsection{Actionable Insights:} Translate the findings from feature importance analysis into actionable business insights. This could involve improving service offerings, optimizing pricing plans, or enhancing customer communication channels based on the most influential churn factors.
    \subsubsection{Explainable AI:} If interpretability of the models is important, consider using techniques like LIME (Local Interpretable Model-Agnostic Explanations) to understand why the models make specific predictions. This can be valuable for building trust and transparency in the churn prediction process.

\subsection{Additional Recommendations:}

    \subsubsection{Data Quality:} Ensure the quality of the data used for training the models. Regularly clean and update the data to maintain model performance.
    \subsubsection{Customer Segmentation:} Segment your customer base based on relevant factors beyond churn risk. This allows for more targeted marketing and retention efforts.
    \subsubsection{Real-time Churn Prediction:} Explore implementing real-time churn prediction to proactively identify and address at-risk customers as their behavior changes.
